\documentclass[12pt]{scrartcl}

\begin{document}
\bibliographystyle{acm}

\title{Project Brief}
\subtitle{Optimise Request for Feedback Schedules to\\
Maximise Response Rates and Quality at SoPost}
\author{James Burton\\SoPost\\jburton@sopost.com}
\maketitle

\section{Background and Motivation}

Over SoPost's 10 years of operation it has gathered a plethora of data regarding
the performance of various sampling campaigns. For context, a typical campaign
follows three stages:

\begin{enumerate}
    \item Claim: Samples are being claimed by consumers.
    \item Feedback: SoPost is gathering feedback from consumers regarding the sample.
    \item Purchase: Emails are sent to drive the consumers to purchase a full product.
\end{enumerate}

One of the key factors that sets SoPost apart from its competitors is the
aforementioned wealth of information, knowledge and experience that allows
SoPost to position itself, not just as a sampling platform, but as a sampling
consultant.

For the purposes of this project, we are interested in data gathered surrounding
the "Feedback" stage of a campaign. When comparing the performance of campaigns,
one metric that gets used frequently is the Request for Feedback (RFF) response rate.
For example if RFFs are sent to 1000 consumers and we receive 70 responses then the
response rate would be 0.07 or 7\%.

SoPost is interested in using its historic data to determine how factors such
as day of the week affect the response rate of RFF emails. As an example, one
might imagine that Mondays are a bad day to send emails since most people are
busy dealing with tasks at the start of a new week or handling issues that arose
over the weekend. Having insights such as this would allow SoPost to advise brands
on the best options when configuring a campaign and further cement SoPost as "the
sampling experts".

The Head of Innovation at SoPost has expressed particular interest in the outcome
of this project and will serve as a key stakeholder throughout its duration.

\section{Project Objectives}
\label{section:objectives}

In this section we shall outline what we hope to achieve with this project by using
the OKR framework and defining our objectives as measured by various key results.\\

\textbf{O1}: Determine the best day of the week for sending RFFs to maximise response rates.
\begin{itemize}
    \item KR1: Identify relevant data from SoPost and any necessary external sources.
    \item KR2: Produce a dashboard in Looker Studio showing average response rates
        per day of the week and providing various filters.
    \item KR3: Train a predictive model which takes into account many factors at once
        such time between sample claim and RFF send.
    \item KR4: Produce a report for key stakeholders.
\end{itemize}

\textbf{O2} (Stretch): Gain similar insights optimising for metrics other than response rate.
\begin{itemize}
    \item KR4: Repeat analysis optimising for review sentiment.
    \item KR5a: Produce an appropriate method for measuring review quality.
    \item KR5b: Repeat analysis optimising for review quality.
\end{itemize}

\section{The Data}

This section defines the data requirements of this project. We have divided
the discussion further into two, firstly discussing the internal data that
we shall be using from SoPost and secondly any external data we shall require.

\subsection{Internal Data}

SoPost uses Google BigQuery as a data warehouse, this shall be our internal data source
for this project. One significant advantage in taking our data from BigQuery is that
we do not store any personal information in there meaning that GDPR concerns have already
been handled for us.

There are two tables we shall be primarily focusing on. The first of theses tables is
\texttt{scheduled\_notifications}
which contains the scheduled send times for both RFF and Request for Purchase (RFP) emails.
The second table is the \texttt{review} table which which can be used to join on other tables
to obtain various details about the product type. The schemas for these tables can be found
in Figures \ref{schema:scheduled_notifications} and \ref{schema:review}.

\begin{figure}
    \centering
    \begin{tabular}{ |c||c|c| }
        \hline
        \textbf{Field} & \textbf{Type} & \textbf{Mode} \\
        \hline
        count & INTEGER & NULLABLE \\
        campaign\_id & STRING & NULLABLE \\
        activity\_id & STRING & NULLABLE \\
        event\_type & STRING & NULLABLE \\
        event\_date & TIMESTAMP & NULLABLE \\
        \hline
    \end{tabular}
    \caption{Schema of the \texttt{scheduled\_notifications} table.}
    \label{schema:scheduled_notifications}
\end{figure}

\begin{figure}
    \centering
    \begin{tabular}{ |c||c|c| }
        \hline
        \textbf{Field} & \textbf{Type} & \textbf{Mode} \\
        \hline
        id & STRING & REQUIRED \\
        order\_id & STRING & REQUIRED \\
        product\_id & STRING & REQUIRED \\
        activity\_id & STRING & REQUIRED \\
        campaign\_id & STRING & NULLABLE \\
        questions & JSON & NULLABLE \\
        created & TIMESTAMP & NULLABLE \\
        cluster & STRING & NULLABLE \\
        \hline
    \end{tabular}
    \caption{Schema of the \texttt{review} table.}
    \label{schema:review}
\end{figure}


% regarding SoPost data we have
%   sample type/size
%   product type
%   sampling platform
%   various demographics about respondants
%   datetimes of RFF emails being sent
%   datetimes of responses
%   can analyse quality of responses (how detailed)
%   can analyse positivity of responses

% timeliness of data is important e.g. lockdowns, bank holdiays etc. think locale too


\subsection{External Data}

While SoPost can provide the majority of the data we will need for our
analysis there are two issues which require the use of external data to solve.
The first is that countries have public holidays which may affect response
rates on the given days. Secondly, during 2020 and 2021 various countries
experienced varying levels of lockdown in response to the COVID-19 pandemic.

We shall decide what to do about these dates later in the project but we
must first obtain data to actually identify when those affected dates were.
With respect to public holidays, Microsoft provides an open dataset \cite{ms:public-hols}.
As for lockdown data, a quick search shows that Kaggle provides the following
dataset \cite{kaggle:lockdowns}.

% Create a view with all the data we need from SoPost
% Load external data into an accessible source e.g. CSV
% Analyse data for cleanliness etc.
% Use Looker studio to produce some basic visualisations with filters
% Research more relevant statistical analysis models or machine learning
    % data mining?
% Produce a predictive model and find which factors are relevant
% Produce a report for stakeholders

\bibliography{refs}

\end{document}